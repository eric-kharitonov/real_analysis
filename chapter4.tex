\documentclass[11pt]{article}

\usepackage{amsfonts}
\usepackage{fancyhdr}
\usepackage{hyperref}
\usepackage{graphicx}
\usepackage{framed}
\usepackage{tabularx}
\usepackage{array}
\usepackage[utf8]{inputenc}
\usepackage{color,soul}
\usepackage[dvipsnames]{xcolor}
\usepackage{amsmath,amsthm,amssymb, mathtools}

\usepackage[margin=3cm, headheight=15pt]{geometry}

\pagestyle{fancyplain}
\lhead{\fancyplain{Eric Kharitonov}{Eric Kharitonov}}
\rhead{\fancyplain{Analysis - chapter 4}{Analysis - chapter 4}}

\usepackage{dcolumn}
\newcolumntype{2}{D{.}{}{2.0}}

\begin{document}
\newcommand{\R}{\mathbb{R}}
\newcommand{\Q}{\mathbb{Q}}
\newcommand{\Z}{\mathbb{Z}}
\newcommand{\N}{\mathbb{N}}
\newcommand\numberthis{\addtocounter{equation}{1}\tag{\theequation}}

\section*{Exercise 4.3}
	
Prove that if $\sum\limits_{k=1}^{\infty}|a_k|$ converges then so does $\sum\limits_{k=1}^{\infty}a_k$.

\bigskip

\paragraph{The proof} ~\\

Let us assume $\sum\limits_{k=1}^{\infty}|a_k| = L$ where L is some real number. ~\\

There are 3 cases to this question:
\begin{enumerate}
	\item $(a_k)$ has finite number of negatives
	\item $(a_k)$ has finite number of positives
	\item $(a_k)$ has infinitely many positives and negatives
\end{enumerate} 

\paragraph{Case 1 - Finite number of negatives} ~\\

Let us assume that $(a_{k_i})$ is the set containing all the negatives. We know it is of a finite size and hence $\sum |a_{k_i}|=\sum\limits_{n \in (k_i)} |a_n|$ is some finite number N. 

\begin{align*}
\sum\limits_{n \in (k_i)} |a_n| &= N\\
-\sum\limits_{n \in (k_i)} a_n &= N\\
\sum\limits_{n \in (k_i)} a_n &= -N \numberthis \label{eq1}
\end{align*}

Because $(k_i)$ is a finite set we can write the following:

\begin{align*}
\sum\limits_{k=1}^{\infty}|a_k| &= \sum\limits_{n \in (k_i)} |a_n| + \sum\limits_{n \notin (k_i)} |a_{n}| = L\\
\sum\limits_{k=1}^{\infty}|a_k| &= N + \sum\limits_{n \notin (k_i)} a_{n} = L\\
\sum\limits_{n \notin (k_i)} a_{n} &= L - N \numberthis \label{eq2}
\end{align*}

Finally, combining \eqref{eq1} and \eqref{eq2}

$\sum\limits_{k=1}^{\infty}a_k = \sum\limits_{n \in (k_i)} a_n + \sum\limits_{n \notin (k_i)} a_{n} = -N + L - N = L - 2N$


and so $\sum\limits_{k=1}^{\infty}a_k$ converges.

\paragraph{Case 2 - Finite number if positives} ~\\

The proof for this case is very similar to Case 1.

\paragraph{Case 3 - Infinite number of negatives and positives} ~\\

Notice that if $\sum\limits_{k=1}^{\infty}|a_k|$ converges, then $\sum\limits_{k=1}^{\infty}|a_k|$ is bounded by some $C>0$. Let $(a_{k_n})$ be the positive subsequence of $(a_k)$, and let $a_{k_m}$ be the negative subsequence of $(a_k)$. Notice that $\sum\limits_{n=1}^{\infty}a_{k_n}$ must also be bounded by C, and since $a_{k_n}>0$ for all n, $\sum\limits_{n=1}^{\infty}a_{k_n}$ converges, and since $\sum\limits_{n=1}^{\infty}a_{k_m}$ is bounded by $-C$ and each $a_{k_m}<0$ then $\sum\limits_{m=1}^{\infty}a_{k_m}$ converges as well. So by limits laws, the series $\sum\limits_{k=1}^{\infty}a_k$ must also converge.


\section*{Exercise 4.13a} 

Prove that if $(ka_k)$ coverges to $L:L\neq0$, then $\sum\limits_{k=1}^{\infty}a_k$ diverges. Give example to show converse is false.

\bigskip

\paragraph{The proof}~\\

We know $(ka_k)\rightarrow L:L\neq0$, so for fix $\epsilon>0$ there exists $N\in\N:$ 
\begin{align*}
	{|ka_k - L| < \epsilon}\\
	{-\epsilon<ka_k-L<\epsilon}\\
	{L-\epsilon<ka_k<\epsilon+L}\\
	{\frac{L-\epsilon}{k}<a_k<\frac{L+\epsilon}{k}}	
\end{align*} 	
For all $k>N$. Without loss of generality, Let $L>0$ and $\epsilon<L$. then the following is true:
\begin{align*}
		{\frac{L-\epsilon}{k}<a_k<\frac{L+\epsilon}{k}}\\
		{0<\frac{L-\epsilon}{k}<a_k<\frac{L+\epsilon}{k}}\\
		{0<\frac{L-\epsilon}{k}<a_k}
\end{align*}
For all $k>N$. And since $\sum\limits_{k=1}^{\infty}\frac{L-\epsilon}{K}$ diverges, then so does $\sum\limits_{k=1}^{\infty}a_k$.
\paragraph{The Converse Example} ~\\

Let $a_k=1$, then $(ka_k)=k\rightarrow\infty$

\section*{Exercise 4.13b}

Prove that if $(k^2 a_k)$ coverges to $L:L\in\R$, then $\sum\limits_{k=1}^{\infty}a_k$ converges. Give example to show converse is false.

\paragraph{The proof} ~\\

 Let us assume $(k^2a_k) \rightarrow L:L\in\R$. `\\
 
 There are 3 cases to this question:
 \begin{enumerate}
 	\item $L>0$
 	\item $L=0$
 	\item $L<0$
 \end{enumerate}

\paragraph{Case 1 - $L>0$} ~\\

We know $(k^2a_k)\rightarrow L:L>0$, so if we fix $L>\epsilon>0$, then there exists $N\in\N:$
\begin{align*}
	|k^2a_k-L|<\epsilon\\
	\frac{L-\epsilon}{k^2}<a_k<\frac{L+\epsilon}{k^2}\\
	0<\frac{L-\epsilon}{k^2}<a_k<\frac{L+\epsilon}{k^2}\\
	0<a_k<\frac{L+\epsilon}{k^2}
\end{align*}
For all $K>N$. Since $\sum\limits_{k=1}^{\infty}\frac{L+\epsilon}{k^2}$ converges, then by comparison law, so does $\sum\limits_{k=1}^{\infty}a_k$

\paragraph{Case 2 - $L<0$} ~\\

This proof is very similar to Case 1.

\paragraph{Case 3 - $L=0$} ~\\

If we prove that $\sum\limits_{k=1}^{\infty}|a_k|$ converges, then so does $\sum\limits_{k=1}^{\infty}a_k$.
Notice that for $\epsilon>0$ there exists N:
\begin{align*}
	{|k^2a_k-L|<\epsilon}\\
	{|K^2a_k|<\epsilon}\\
	{-\epsilon<k^2a_k<\epsilon}\\
	{-\frac{\epsilon}{k^2}}<a_k<\frac{\epsilon}{k^2}
\end{align*}
so:
\begin{align*}
	0<|a_k|<\frac{\epsilon}{k^2}
\end{align*}
for all $k>N$. Since $\sum\limits_{k=1}^{\infty}\frac{\epsilon}{k^2}$ converges, by comparison test, so does $\sum\limits_{k=1}^{\infty}|a_k|$, which implies that $\sum\limits_{k=1}^{\infty}a_k$ converges as well.

\paragraph{The converse example} ~\\ 

Same as part a.

\section*{Exercie 15}
Prove that if $(a_k)$ is a decreasing sequence and converges to 0, then $\sum\limits_{k=1}^{\infty}a_k$ converges if and only if $\sum\limits_{k=1}^{\infty}2^k a_{2^k}$ converges.
\paragraph{Pf.}
For the backward direction notice that:
\begin{align*}
	\sum\limits_{k=1}^{\infty}a_k &= a_1 + a_2 + a_3 + a_4 + a_5 + a_6 + a_7 + \dots\\
	&\leq a_1 + a_2 + a_2 + a_4 + a_4 + a_4 + a_4 + \dots\\
	&= a_1 + \sum\limits_{k=1}^{\infty}2^k a_{2^k}
\end{align*}
and since we assumed that $ \sum\limits_{k=1}^{\infty}2^k a_{2^k}$ converges, then so does $a_1 + \sum\limits_{k=1}^{\infty}2^k a_{2^k}$, which by comparison test, follows that $\sum\limits_{k=1}^{\infty}a_k$ converges.\\
Now for the foreward direction.

\end{document}